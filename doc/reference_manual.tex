\documentclass[11pt,a4paper]{article}
\usepackage[utf8]{inputenc}
\usepackage[T1]{fontenc}
\usepackage{amsmath,amsfonts,amssymb}
\usepackage{graphicx}
\usepackage{hyperref}
\usepackage{listings}
\usepackage{xcolor}
\usepackage{geometry}
\usepackage{fancyvrb}
\geometry{margin=1in}

\definecolor{codegreen}{rgb}{0,0.6,0}
\definecolor{codegray}{rgb}{0.5,0.5,0.5}
\definecolor{codepurple}{rgb}{0.58,0,0.82}
\definecolor{backcolour}{rgb}{0.95,0.95,0.92}

\lstdefinestyle{swiftstyle}{
    backgroundcolor=\color{backcolour},
    commentstyle=\color{codegreen},
    keywordstyle=\color{magenta},
    numberstyle=\tiny\color{codegray},
    stringstyle=\color{codepurple},
    basicstyle=\ttfamily\footnotesize,
    breakatwhitespace=false,
    breaklines=true,
    captionpos=b,
    keepspaces=true,
    numbers=left,
    numbersep=5pt,
    showspaces=false,
    showstringspaces=false,
    showtabs=false,
    tabsize=2,
    language=Swift
}

\title{Diet Solver Hearing Health Integration\\
Reference Manual}
\author{Shyamal Chandra}
\date{\today}

\begin{document}

\maketitle

\tableofcontents
\newpage

\section{Introduction}

This reference manual provides comprehensive documentation for the hearing health integration in the Diet Solver system. It includes API references, usage examples, and best practices for working with hearing data models and exercise recommendations.

\section{Data Models}

\subsection{HearingPrescription}

The \texttt{HearingPrescription} model represents professional hearing assessments.

\subsubsection{Structure}

\begin{lstlisting}[style=swiftstyle]
struct HearingPrescription: Codable {
    var date: Date
    var expirationDate: Date?
    var rightEar: EarPrescription
    var leftEar: EarPrescription
    var hearingAidSettings: HearingAidSettings?
    var notes: String?
}
\end{lstlisting}

\subsubsection{Example Usage}

\begin{lstlisting}[style=swiftstyle]
// Create a hearing prescription
let prescription = HearingPrescription(
    date: Date(),
    expirationDate: Calendar.current.date(byAdding: .year, value: 1, to: Date()),
    rightEar: HearingPrescription.EarPrescription(
        hearingLossType: .sensorineural,
        hearingLossSeverity: .mild
    ),
    leftEar: HearingPrescription.EarPrescription(
        hearingLossType: .sensorineural,
        hearingLossSeverity: .mild
    ),
    notes: "Regular monitoring recommended"
)

// Store in health data
healthData.hearingPrescription = prescription
\end{lstlisting}

\subsection{DailyAudioHearingTest}

The \texttt{DailyAudioHearingTest} model records daily self-administered hearing tests.

\subsubsection{Structure}

\begin{lstlisting}[style=swiftstyle]
struct DailyAudioHearingTest: Codable, Identifiable {
    let id: UUID
    var date: Date
    var time: Date
    var rightEar: EarTest
    var leftEar: EarTest
    var bothEars: BothEarsTest
    var testType: TestType
    var device: TestDevice
    var environment: TestEnvironment
    var notes: String?
}
\end{lstlisting}

\subsubsection{Example Usage}

\begin{lstlisting}[style=swiftstyle]
// Create a quick hearing test
var test = DailyAudioHearingTest(
    date: Date(),
    time: Date(),
    testType: .quick,
    device: .iphone,
    environment: DailyAudioHearingTest.TestEnvironment(
        backgroundNoise: .quiet,
        roomAcoustics: .normal
    )
)

// Add pure tone threshold for right ear
test.rightEar.pureToneThresholds = [
    DailyAudioHearingTest.EarTest.FrequencyThreshold(
        frequency: 1000,  // Hz
        threshold: 20.0,   // dB HL
        ear: .right
    ),
    DailyAudioHearingTest.EarTest.FrequencyThreshold(
        frequency: 2000,
        threshold: 25.0,
        ear: .right
    )
]

// Check for tinnitus
test.rightEar.tinnitusPresence = true
test.rightEar.tinnitusSeverity = .mild

// Add to health data
healthData.dailyAudioHearingTests.append(test)
\end{lstlisting}

\subsubsection{Test Types}

\begin{itemize}
    \item \texttt{.pureTone}: Pure tone audiometry
    \item \texttt{.speech}: Speech recognition test
    \item \texttt{.word}: Word recognition test
    \item \texttt{.comprehensive}: Full hearing assessment
    \item \texttt{.quick}: Quick daily check
\end{itemize}

\subsection{MusicHearingSession}

The \texttt{MusicHearingSession} model tracks music listening activities.

\subsubsection{Structure}

\begin{lstlisting}[style=swiftstyle]
struct MusicHearingSession: Codable, Identifiable {
    let id: UUID
    var date: Date
    var startTime: Date
    var endTime: Date?
    var duration: Double  // minutes
    var musicType: MusicType
    var genre: String?
    var volumeLevel: VolumeLevel
    var device: MusicDevice
    var listeningMode: ListeningMode
    var hearingProtection: Bool
    var hearingFatigue: HearingFatigueLevel?
    var enjoyment: EnjoymentLevel
    var notes: String?
}
\end{lstlisting}

\subsubsection{Example Usage}

\begin{lstlisting}[style=swiftstyle]
// Create a music listening session
let startTime = Date()
let session = MusicHearingSession(
    date: startTime,
    startTime: startTime,
    duration: 30.0,  // 30 minutes
    musicType: .classical,
    genre: "Baroque",
    volumeLevel: .moderate,
    device: .airpodsPro,
    listeningMode: .active,
    hearingProtection: false,
    hearingFatigue: .none,
    enjoyment: .high,
    notes: "Morning listening session"
)

// Add to health data
healthData.musicHearingSessions.append(session)
\end{lstlisting}

\subsubsection{Music Types}

\begin{itemize}
    \item \texttt{.classical}, \texttt{.jazz}, \texttt{.rock}, \texttt{.pop}
    \item \texttt{.electronic}, \texttt{.world}, \texttt{.instrumental}
    \item \texttt{.vocal}, \texttt{.therapeutic}, \texttt{.binaural}
    \item \texttt{.nature}, \texttt{.meditation}, \texttt{.other}
\end{itemize}

\subsubsection{Listening Modes}

\begin{itemize}
    \item \texttt{.active}: Focused, attentive listening
    \item \texttt{.background}: Background music
    \item \texttt{.focused}: Deep concentration
    \item \texttt{.therapeutic}: Therapeutic/healing focus
    \item \texttt{.exercise}: During physical activity
    \item \texttt{.relaxation}: For relaxation
\end{itemize}

\section{Integration with HealthData}

\subsection{Adding Hearing Data}

\begin{lstlisting}[style=swiftstyle]
// Initialize health data
var healthData = HealthData(
    age: 30,
    gender: .male,
    weight: 75.0,
    height: 175.0,
    activityLevel: .moderate
)

// Add hearing prescription
healthData.hearingPrescription = HearingPrescription(...)

// Add daily tests
healthData.dailyAudioHearingTests.append(DailyAudioHearingTest(...))

// Add music sessions
healthData.musicHearingSessions.append(MusicHearingSession(...))
\end{lstlisting}

\subsection{Accessing Hearing Data}

\begin{lstlisting}[style=swiftstyle]
// Get latest hearing test
if let latestTest = healthData.dailyAudioHearingTests.last {
    print("Latest test date: \(latestTest.date)")
    if let threshold = latestTest.rightEar.pureToneThresholds?.first?.threshold {
        print("Right ear threshold: \(threshold) dB HL")
    }
}

// Get recent music sessions
let recentSessions = healthData.musicHearingSessions.filter { session in
    Calendar.current.dateInterval(of: .weekOfYear, for: Date())?
        .contains(session.date) ?? false
}
print("Sessions this week: \(recentSessions.count)")

// Calculate average session duration
let avgDuration = healthData.musicHearingSessions
    .map { $0.duration }
    .reduce(0, +) / Double(max(1, healthData.musicHearingSessions.count))
print("Average duration: \(avgDuration) minutes")
\end{lstlisting}

\section{Exercise Planner Integration}

\subsection{Generating Weekly Plans}

\begin{lstlisting}[style=swiftstyle]
// Create exercise planner
let planner = ExercisePlanner()

// Define exercise goals
let goals = ExerciseGoals(
    weeklyCardioMinutes: 150,
    weeklyStrengthSessions: 2,
    weeklyFlexibilityMinutes: 60,
    weeklyMindBodyMinutes: 120
)

// Generate weekly plan with hearing activities
let plan = planner.generateWeeklyPlan(for: healthData, goals: goals)

// Access hearing-related activities
for dayPlan in plan.weeklyPlan {
    let hearingActivities = dayPlan.activities.filter { activity in
        activity.activity.name.contains("Music") ||
        activity.activity.name.contains("Hearing") ||
        activity.activity.name.contains("Binaural") ||
        activity.activity.name.contains("Nature Sounds")
    }
    
    if !hearingActivities.isEmpty {
        print("Day \(dayPlan.dayOfWeek) has \(hearingActivities.count) hearing activities")
    }
}
\end{lstlisting}

\subsection{Getting Recommendations}

\begin{lstlisting}[style=swiftstyle]
// Get exercise recommendations based on hearing data
let recommendations = planner.recommendExercises(
    for: healthData,
    dayOfWeek: 0  // Sunday
)

// Filter hearing-related recommendations
let hearingRecommendations = recommendations.filter { activity in
    activity.name.contains("Music") ||
    activity.name.contains("Hearing") ||
    activity.name.contains("Binaural") ||
    activity.name.contains("Nature Sounds")
}

// Display recommendations
for activity in hearingRecommendations {
    print("Recommended: \(activity.name)")
    print("Benefits: \(activity.benefits.joined(separator: ", "))")
}
\end{lstlisting}

\subsection{Focus Areas}

\begin{lstlisting}[style=swiftstyle]
// Generate plan to see focus areas
let plan = planner.generateWeeklyPlan(for: healthData, goals: goals)

// Check for hearing-related focus areas
let hearingFocusAreas = plan.focusAreas.filter { area in
    area.contains("Hearing") || area.contains("Music")
}

if !hearingFocusAreas.isEmpty {
    print("Hearing focus areas:")
    for area in hearingFocusAreas {
        print("  - \(area)")
    }
}
\end{lstlisting}

\section{Exercise Database}

\subsection{Accessing Hearing Activities}

\begin{lstlisting}[style=swiftstyle]
// Get exercise database
let database = ExerciseDatabase.shared

// Find all hearing-related activities
let hearingActivities = database.activities.filter { activity in
    activity.name.contains("Music") ||
    activity.name.contains("Hearing") ||
    activity.name.contains("Binaural") ||
    activity.name.contains("Nature Sounds") ||
    activity.name.contains("Audio Book")
}

// Display available activities
for activity in hearingActivities {
    print("\(activity.name):")
    print("  Category: \(activity.category)")
    print("  Intensity: \(activity.intensity)")
    print("  Benefits: \(activity.benefits.joined(separator: ", "))")
}
\end{lstlisting}

\subsection{Available Hearing Activities}

\begin{itemize}
    \item \textbf{Music Listening Session}: General music listening for hearing health
    \item \textbf{Active Music Listening}: Focused, attentive music listening
    \item \textbf{Hearing Exercise}: Specific exercises for hearing health
    \item \textbf{Nature Sounds Therapy}: Therapeutic nature sounds
    \item \textbf{Binaural Beats Session}: Binaural beats for focus and meditation
    \item \textbf{Audio Book Listening}: Educational audio content
\end{itemize}

\section{Best Practices}

\subsection{Regular Testing}

\begin{lstlisting}[style=swiftstyle]
// Perform daily hearing test
func performDailyHearingTest() {
    let test = DailyAudioHearingTest(
        date: Date(),
        testType: .quick,
        device: .iphone,
        environment: DailyAudioHearingTest.TestEnvironment(
            backgroundNoise: .quiet,
            roomAcoustics: .normal
        )
    )
    
    // Add test results
    // ... configure test results ...
    
    healthData.dailyAudioHearingTests.append(test)
}
\end{lstlisting}

\subsection{Music Session Tracking}

\begin{lstlisting}[style=swiftstyle]
// Track music listening session
func trackMusicSession(
    duration: Double,
    musicType: MusicHearingSession.MusicType,
    volumeLevel: MusicHearingSession.VolumeLevel
) {
    let session = MusicHearingSession(
        date: Date(),
        startTime: Date(),
        duration: duration,
        musicType: musicType,
        volumeLevel: volumeLevel,
        device: .airpodsPro,
        listeningMode: .active,
        hearingProtection: false
    )
    
    healthData.musicHearingSessions.append(session)
}
\end{lstlisting}

\subsection{Monitoring Hearing Health}

\begin{lstlisting}[style=swiftstyle]
// Monitor hearing health trends
func analyzeHearingHealth() {
    // Check test frequency
    let recentTests = healthData.dailyAudioHearingTests.filter { test in
        Calendar.current.dateInterval(of: .weekOfYear, for: Date())?
            .contains(test.date) ?? false
    }
    
    if recentTests.count < 3 {
        print("Warning: Low test frequency. Consider daily tests.")
    }
    
    // Check for hearing loss indicators
    if let latestTest = healthData.dailyAudioHearingTests.last {
        if let threshold = latestTest.rightEar.pureToneThresholds?.first?.threshold {
            if threshold > 25 {
                print("Warning: Possible hearing loss detected (\(threshold) dB HL)")
            }
        }
    }
    
    // Check for tinnitus
    if let latestTest = healthData.dailyAudioHearingTests.last {
        if latestTest.rightEar.tinnitusPresence == true ||
           latestTest.leftEar.tinnitusPresence == true {
            print("Tinnitus detected. Consider therapeutic activities.")
        }
    }
}
\end{lstlisting}

\section{Data Persistence}

\subsection{Encoding and Decoding}

\begin{lstlisting}[style=swiftstyle]
// Encode health data with hearing information
func saveHealthData(_ healthData: HealthData) throws {
    let encoder = JSONEncoder()
    encoder.dateEncodingStrategy = .iso8601
    encoder.outputFormatting = .prettyPrinted
    
    let data = try encoder.encode(healthData)
    try data.write(to: fileURL)
}

// Decode health data
func loadHealthData() throws -> HealthData {
    let data = try Data(contentsOf: fileURL)
    let decoder = JSONDecoder()
    decoder.dateDecodingStrategy = .iso8601
    
    return try decoder.decode(HealthData.self, from: data)
}
\end{lstlisting}

\section{Troubleshooting}

\subsection{Common Issues}

\subsubsection{Missing Hearing Activities}

If hearing activities don't appear in recommendations:

\begin{lstlisting}[style=swiftstyle]
// Verify activities exist in database
let database = ExerciseDatabase.shared
let musicActivity = database.activities.first { $0.name == "Music Listening Session" }
assert(musicActivity != nil, "Music Listening Session should exist")
\end{lstlisting}

\subsubsection{Recommendations Not Appearing}

If recommendations don't match expected behavior:

\begin{lstlisting}[style=swiftstyle]
// Check health data has hearing information
if healthData.dailyAudioHearingTests.isEmpty {
    print("No hearing tests found. Recommendations may be limited.")
}

// Verify planner is using latest health data
let recommendations = planner.recommendExercises(
    for: healthData,
    dayOfWeek: 0
)
print("Recommendations count: \(recommendations.count)")
\end{lstlisting}

\section{API Reference Summary}

\subsection{Key Types}

\begin{itemize}
    \item \texttt{HearingPrescription}: Professional hearing assessment
    \item \texttt{DailyAudioHearingTest}: Daily self-administered test
    \item \texttt{MusicHearingSession}: Music listening activity
    \item \texttt{HearingHealthSummary}: Summary of hearing health
    \item \texttt{HearingAnalysisReport}: Comprehensive analysis report
\end{itemize}

\subsection{Key Methods}

\begin{itemize}
    \item \texttt{ExercisePlanner.generateWeeklyPlan(for:goals:)}: Generate weekly plan
    \item \texttt{ExercisePlanner.recommendExercises(for:dayOfWeek:)}: Get recommendations
    \item \texttt{ExerciseDatabase.shared}: Access exercise database
\end{itemize}

\section{Examples}

\subsection{Complete Workflow}

\begin{lstlisting}[style=swiftstyle]
// 1. Initialize health data
var healthData = HealthData(
    age: 30,
    gender: .male,
    weight: 75.0,
    height: 175.0,
    activityLevel: .moderate
)

// 2. Perform daily hearing test
let test = DailyAudioHearingTest(
    date: Date(),
    testType: .quick,
    device: .iphone
)
healthData.dailyAudioHearingTests.append(test)

// 3. Track music session
let session = MusicHearingSession(
    date: Date(),
    duration: 30.0,
    musicType: .classical,
    volumeLevel: .moderate
)
healthData.musicHearingSessions.append(session)

// 4. Generate exercise plan
let planner = ExercisePlanner()
let goals = ExerciseGoals()
let plan = planner.generateWeeklyPlan(for: healthData, goals: goals)

// 5. Get recommendations
let recommendations = planner.recommendExercises(
    for: healthData,
    dayOfWeek: 0
)

// 6. Display results
print("Weekly plan has \(plan.weeklyPlan.count) days")
print("Focus areas: \(plan.focusAreas.joined(separator: ", "))")
print("Recommendations: \(recommendations.count) activities")
\end{lstlisting}

\section{Conclusion}

This reference manual provides comprehensive documentation for working with hearing health integration in the Diet Solver system. For additional information, refer to the complete paper and source code documentation.

\end{document}
